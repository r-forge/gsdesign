\section{Basic Features\label{sec:basicfeatures}}

There are several key design features common to \texttt{gsDesign()},
\texttt{gsProbability()}, and \texttt{gsCP()}:

\begin{enumerate}
\item Computations are based on asymptotic approximations as provided by
Jennison and Turnbull \cite{JTBook}.

\item Power plots and boundary plots are available, in addition to various
printing formats for summarization.
\end{enumerate}

In addition, the following apply to \texttt{gsDesign()}:

\begin{enumerate}
\item Rather than supporting a wide variety of endpoint or design types (e.g.,
normal, binomial, time to event), the \texttt{gsDesign()} routine allows input
of the sample size for a fixed design with no interim analysis and adjusts the
sample size appropriately for a group sequential design.

\item Two-sided symmetric and asymmetric designs are supported, as well as
one-sided designs.

\item The spending function approach to group sequential design first
published by Lan and DeMets \cite{LanDeMets} is implemented. Commonly used
spending functions published by Hwang, Shih, and DeCani \cite{HwangShihDeCani}
and by Kim and DeMets \cite{KimDeMets} are provided. Other built-in spending
functions are included. Two- and three-parameter spending functions are
particularly flexible. There is also point-wise specification of spending
available. Finally, specifications are given for users to write their own
spending functions.

\item As an alternative to the spending function approach, the Wang and
Tsiatis \cite{WangTsiatis} family of boundaries is also available for
symmetric or one-sided designs. This family includes O'Brien-Fleming and
Pocock boundaries as members.

\item For asymmetric designs, lower bound spending functions may be used to
specify lower boundary crossing probabilities under the alternative hypothesis
(beta spending) or null hypothesis (recommended when number of analyses is
large or when faster computing is required---e.g., for optimization).

\item Normally it is assumed that when a boundary is crossed at the time of an
analysis, the clinical trial must stop without a positive finding. In this
case, the boundary is referred to as binding. For asymmetric designs, a user
option is available to ignore lower bounds when computing Type I error. Under
this assumption the lower bound is referred to as non-binding. That is, the
trial may continue rather than absolutely requiring stopping when the lower
bound is crossed. This is a conservative design option sometimes requested by
regulators to preserve Type I error when they assume a sponsor may choose to
ignore an aggressive futility (lower) bound if it is crossed.
\end{enumerate}

