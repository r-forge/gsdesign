\section{Deriving group sequential designs \label{sec:gsDesign}}
There are many ways to specify a group sequential design to obtain a desired power and Type I error. 
For planning purposes, the number, $k$ and relative timing $0<t_1<...<t_k=1$ of interim analyses are fixed.
Given these values, there are two general approaches to deriving boundaries for a group sequential trial:
\begin{itemize}
\item {\bf The error spending approach}. Specify boundary crossing probabilities at each analysis and derive a sample size and boundary values based on these values.
This is most commonly done with the error spending function approach proposed by Lan and DeMets \cite{LanDeMets}, which is discussed at some length in Section \ref{sec:spendfn}. 
We present this method in brief in Section \ref{sec:errspend} and follow this with simple examples.
\item {\bf The boundary family approach.} Specify how big boundary values should be relative to each other and adjust these relative values by a constant multiple to control overall error rates. Sample size adjustment is also part of this derivation. The commonly applied boundary family approach uses the Wang-Tsiatis \cite{WangTsiatis} family which includes bounds by Pocock \cite{PocockBound} and O'Brien and Fleming \cite{OF}. This will be discussed in Section \ref{sec:boundfam}.
\end{itemize}

\subsection{Boundary derivation using boundary crossing probabilities \label{sec:errspend}}
\subsubsection{Types of error probabilities used: \texttt{test.type}}
Before starting a discussion of spending functions, different methods of computing Type I error are discussed.
Boundary crossing probabilities for upper bounds may be specified to \texttt{gsDesign()} using either $\alpha_i(0)$ or $\alpha_i^+(0)$, $i=1,2,\ldots,k$.
In the first case, it is assumed that a trial stops when either a lower or upper bound is crossed and the only Type I error occurs when the first time a boundary is crossed it is an upper bound.
In the second case, it is assumed that a lower bound may be ignored if crossed and the Type I error is the probability of ever crossing an upper bound if the trial is never stopped for crossing a lower bound.
As we have seen, the difference between these boundary crossing probabilities may be small.
The differences can be meaningful, however, when agressive futility bounds are employed to require, say, an early positive treatment effect trend.

For lower bounds, either $\beta_i(\delta)$ or $\beta_i(0)$, $i=1,2,\ldots,k$, may be specified. 

Sample size and boundaries that have appropriate boundary crossing probabilities and power are derived numerically using computational methods given in detail in Chapter 19 of Jennison and Turnbull \cite{JTBook}. The \texttt{gsDesign()} parameter \texttt{test.type} specifies which boundary crossing probabilities are used as outlined in Table \ref{tab:testtype}.
\begin{table}[t]
\begin{center}
\caption{Boundary crossing probabilities used to set boundaries in \texttt{gsDesign()} by \texttt{test.type}.}\label{tab:testtype}
\begin{tabular}[c]{ccc}
\texttt{test.type}&Upper bound & Lower bound\\\hline
1& $\alpha_i^{+}(0)$ & None\\
2&$\alpha(0)$ & $\beta_i(0)$\\
3&$\alpha_i(0)$ & $\beta_i(\delta)$\\
4&$\alpha_i^{+}(0)$ & $\beta_i(\delta)$\\
5&$\alpha(0)$ & $\beta_i(0)$\\
6&$\alpha^{+}(0)$ & $\beta_i(0)$
\end{tabular}
\end{center}
\end{table}

For \texttt{test.type}=1, 2 and 5, boundaries can be computed in a single step
just by knowing the cumulative proportion of the final planned statistical information (sample size/number of events) at each analysis that is specified using the \texttt{timing} input variable. 
For \texttt{test.type}=6, the upper and lower boundaries are computed separately and independently using these same methods. For \texttt{test.type}=1, 2, 5 or 6 the total sample size is then set to obtain the desired power under the alternative hypothesis by using a root finding algorithm.

For \texttt{test.type}=3 and 4 sample size and bounds are set
simultaneously using an iterative algorithm. 
This computation is slightly more complex than the above. 
This does not make any noticeable difference in normal use of the \texttt{gsDesign()}. 
However, for user-developed routines that require repeated calls to \texttt{gsDesign()} ({\it e.g.}, finding an optimal design), there may be noticeably slower performance when test.type=3 or 4 is used.
