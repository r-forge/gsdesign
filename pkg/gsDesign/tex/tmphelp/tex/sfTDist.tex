\HeaderA{sfTDist}{4.7: t-distribution Spending Function}{sfTDist}
\keyword{design}{sfTDist}
\begin{Description}\relax
The function \code{sfTDist()} provides perhaps the maximum flexibility among spending functions
provided in the \code{gsDesign} package.
This function allows fitting of three points on a cumulative spending function curve; in this case, six parameters
are specified indicating an x and a y coordinate for each of 3 points.
Normally this function will be passed to \code{gsDesign()} in the parameter 
\code{sfu} for the upper bound or
\code{sfl} for the lower bound to specify a spending function family for a design.
In this case, the user does not need to know the calling sequence.
The calling sequence is useful, however, when the user wishes to plot a spending function as demonstrated below
in examples.
\end{Description}
\begin{Usage}
\begin{verbatim}
sfTDist(alpha, t, param)
\end{verbatim}
\end{Usage}
\begin{Arguments}
\begin{ldescription}
\item[\code{alpha}] Real value \eqn{> 0}{} and no more than 1. Normally, 
\code{alpha=0.025} for one-sided Type I error specification
or \code{alpha=0.1} for Type II error specification. However, this could be set to 1 if for descriptive purposes
you wish to see the proportion of spending as a function of the proportion of sample size/information.
\item[\code{t}] A vector of points with increasing values from 0 to 1, inclusive. Values of the proportion of 
sample size/information for which the spending function will be computed.
\item[\code{param}] In the three-parameter specification, the first paramater (a) may be any real value, the second (b) any positive value,
and the third parameter (df=degrees of freedom) any real value 1 or greater. When \code{gsDesign()} is called with a t-distribution
spending function, this is the parameterization printed.
The five parameter specification is \code{c(t1,t2,u1,u2,df)}
where the objective is that the resulting cumulative proportion of spending at \code{t} represented by \code{sf(t)} satisfies
\code{sf(t1)=alpha*u1}, \code{sf(t2)=alpha*u2}. The t-distribution used has \code{df} degrees of freedom.
In this parameterization, all the first four values must be between 0 and 1 and \code{t1 < t2}, \code{u1 < u2}.
The final parameter is any real value of 1 or more. This parameterization can fit any two points satisfying these requirements.
The six parameter specification attempts to fit 3 points, but does not have flexibility to fit any three points.
In this case, the specification for \code{param} is c(t1,t2,t3,u1,u2,u3) where the objective is that \code{sf(t1)=alpha*u1},
\code{sf(t2)=alpha*u2}, and \code{sf(t3)=alpha*u3}.
See examples to see what happens when points are specified that cannot be fit.
\end{ldescription}
\end{Arguments}
\begin{Details}\relax
The t-distribution spending function takes the form
\deqn{f(t;\alpha)=\alpha F(a+bF^{-1}(t))}{}
where \eqn{F()}{} is a cumulative t-distribution function with \code{df} degrees of freedom
and \eqn{F^{-1}()}{} is its inverse.
\end{Details}
\begin{Value}
An object of type \code{spendfn}. See spending functions for further details.
\end{Value}
\begin{Note}\relax
The manual is not linked to this help file, but is available in library/gsdesign/doc/gsDesignManual.pdf
in the directory where R is installed.
\end{Note}
\begin{Author}\relax
Keaven Anderson \email{keaven\_anderson@merck.com}
\end{Author}
\begin{References}\relax
Jennison C and Turnbull BW (2000), \emph{Group Sequential Methods with Applications to Clinical Trials}.
Boca Raton: Chapman and Hall.
\end{References}
\begin{SeeAlso}\relax
\LinkA{Spending function overview}{Spending function overview}, \code{\LinkA{gsDesign}{gsDesign}}, \LinkA{gsDesign package overview}{gsDesign package overview}
\end{SeeAlso}
\begin{Examples}
\begin{ExampleCode}
# 3-parameter specification: a,  b,  df
sfTDist(1, 1:5/6, c(-1, 1.5, 4))$spend

# 5-parameter specification fits 2 points,  in this case
# the 1st 2 interims are at 25% and 50% of observations with
# cumulative error spending of 10% and 20%, respectively
# final parameter is df
sfTDist(1, 1:3/4, c(.25, .5, .1, .2, 4))$spend

# 6-parameter specification fits 3 points
# Interims are at 25%. 50% and 75% of observations
# with cumulative spending of 10%, 20% and 50%, respectively
# Note: not all 3 point combinations can be fit
sfTDist(1, 1:3/4, c(.25, .5, .75, .1, .2, .5))$spend

# Example of error message when the 3-points specified 
# in the 6-parameter version cannot be fit
try(sfTDist(1, 1:3/4, c(.25, .5, .75, .1, .2, .3))$errmsg)

# sfCauchy (sfTDist with 1 df) and sfNormal (sfTDist with infinite df)
# show the limits of what sfTdist can fit 
# for the third point are u3 from 0.344 to 0.6 when t3=0.75
sfNormal(1, 1:3/4, c(.25, .5, .1, .2))$spend[3]
sfCauchy(1, 1:3/4, c(.25, .5, .1, .2))$spend[3]

# plot a few t-distribution spending functions fitting
# t=0.25, .5 and u=0.1, 0.2
# to demonstrate the range of flexibility
t <- 0:100/100
plot(t, sfTDist(0.025, t, c(.25, .5, .1, .2, 1))$spend, 
    xlab="Proportion of final sample size", 
    ylab="Cumulative Type I error spending", 
    main="t-Distribution Spending Function Examples", type="l")
lines(t, sfTDist(0.025, t, c(.25, .5, .1, .2, 1.5))$spend, lty=2)
lines(t, sfTDist(0.025, t, c(.25, .5, .1, .2, 3))$spend, lty=3)
lines(t, sfTDist(0.025, t, c(.25, .5, .1, .2, 10))$spend, lty=4)
lines(t, sfTDist(0.025, t, c(.25, .5, .1, .2, 100))$spend, lty=5)
legend(x=c(.0, .3), y=.025*c(.7, 1), lty=1:5, 
    legend=c("df = 1", "df = 1.5", "df = 3", "df = 10", "df = 100"))
\end{ExampleCode}
\end{Examples}

