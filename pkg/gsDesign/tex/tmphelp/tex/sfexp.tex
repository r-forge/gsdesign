\HeaderA{sfExponential}{4.3: Exponential Spending Function}{sfExponential}
\keyword{design}{sfExponential}
\begin{Description}\relax
The function \code{sfExponential} implements the exponential spending function (Anderson and Clark, 2009).
Normally \code{sfExponential} will be passed to \code{gsDesign} in the parameter \code{sfu} for the upper bound or
\code{sfl} for the lower bound to specify a spending function family for a design.
In this case, the user does not need to know the calling sequence.
The calling sequence is useful, however, when the user wishes to plot a spending function as demonstrated below
in examples.
\end{Description}
\begin{Usage}
\begin{verbatim}
sfExponential(alpha, t, param)
\end{verbatim}
\end{Usage}
\begin{Arguments}
\begin{ldescription}
\item[\code{alpha}] Real value \eqn{> 0}{} and no more than 1. Normally, 
\code{alpha=0.025} for one-sided Type I error specification
or \code{alpha=0.1} for Type II error specification. However, this could be set to 1 if for descriptive purposes
you wish to see the proportion of spending as a function of the proportion of sample size/information.
\item[\code{t}] A vector of points with increasing values from 0 to 1, inclusive. Values of the proportion of 
sample size/information for which the spending function will be computed.
\item[\code{param}] A single positive value specifying the nu parameter for which the exponential spending is to be computed; allowable range is (0, 1.5].
\end{ldescription}
\end{Arguments}
\begin{Details}\relax
An  exponential spending function is defined for any positive \code{nu} and \eqn{0\le t\le 1}{} as
\deqn{f(t;\alpha,\nu)=\alpha(t)=\alpha^{t^{-\nu}}.}{f(t;alpha,nu)=alpha^(t^(-nu)).}
A value of \code{nu=0.8} approximates an O'Brien-Fleming spending function well.

The general class of spending functions this family is derived from requires a continuously increasing 
cumulative distribution function defined for \eqn{x>0}{} and is defined as
\deqn{f(t;\alpha, \nu)=1-F\left(F^{-1}(1-\alpha)/ t^\nu\right).}{f(t; alpha, nu)=1-F(F^(-1)(1-alpha)/ t^nu).}
The exponential spending function can be derived by letting \eqn{F(x)=1-\exp(-x)}{}, the exponential cumulative distribution function.
This function was derived as a generalization of the Lan-DeMets (1983) spending function used to approximate an
O'Brien-Fleming spending function (\code{sfLDOF()}),
\deqn{f(t; \alpha)=2-2\Phi \left( \Phi^{-1}(1-\alpha/2)/ t^{1/2} \right).}{f(t; alpha)=2-2*Phi(Phi^(-1)(1-alpha/2)/t^(1/2)).}
\end{Details}
\begin{Value}
An object of type \code{spendfn}.
\end{Value}
\begin{Note}\relax
The manual shows how to use \code{sfExponential()} to closely approximate an O'Brien-Fleming design.
An example is given below.
The manual is not linked to this help file, but is available in library/gsdesign/doc/gsDesignManual.pdf
in the directory where R is installed.
\end{Note}
\begin{Author}\relax
Keaven Anderson \email{keaven\_anderson@merck.com}
\end{Author}
\begin{References}\relax
Jennison C and Turnbull BW (2000), \emph{Group Sequential Methods with Applications to Clinical Trials}.
Boca Raton: Chapman and Hall.

Lan, KKG and DeMets, DL (1983), Discrete sequential boundaries for clinical trials. \emph{Biometrika}; 70:659-663.
\end{References}
\begin{SeeAlso}\relax
\LinkA{Spending function overview}{Spending function overview}, \code{\LinkA{gsDesign}{gsDesign}}, \LinkA{gsDesign package overview}{gsDesign package overview}
\end{SeeAlso}
\begin{Examples}
\begin{ExampleCode}
# use 'best' exponential approximation for k=6 to O'Brien-Fleming design
# (see manual for details)
gsDesign(k=6, sfu=sfExponential, sfupar=0.7849295,
         test.type=2)$upper$bound

# show actual O'Brien-Fleming bound
gsDesign(k=6, sfu="OF", test.type=2)$upper$bound

# show Lan-DeMets approximation
# (not as close as sfExponential approximation)
gsDesign(k=6, sfu=sfLDOF, test.type=2)$upper$bound

# plot exponential spending function across a range of values of interest
t <- 0:100/100
plot(t, sfExponential(0.025, t, 0.8)$spend,
   xlab="Proportion of final sample size", 
   ylab="Cumulative Type I error spending", 
   main="Exponential Spending Function Example", type="l")
lines(t, sfExponential(0.025, t, 0.5)$spend, lty=2)
lines(t, sfExponential(0.025, t, 0.3)$spend, lty=3)
lines(t, sfExponential(0.025, t, 0.2)$spend, lty=4)
lines(t, sfExponential(0.025, t, 0.15)$spend, lty=5)
legend(x=c(.0, .3), y=.025*c(.7, 1), lty=1:5, 
    legend=c("nu = 0.8", "nu = 0.5", "nu = 0.3", "nu = 0.2",
             "nu = 0.15"))
text(x=.59, y=.95*.025, labels="<--approximates O'Brien-Fleming")
\end{ExampleCode}
\end{Examples}

