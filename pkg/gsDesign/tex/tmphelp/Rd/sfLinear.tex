\HeaderA{sfLinear}{4.6: Piecewise Linear Spending Function}{sfLinear}
\keyword{design}{sfLinear}
\begin{Description}\relax
The function \code{sfLinear()} allows specification of a piecewise linear spending function.
This provides complete flexibility in setting spending at desired timepoints in a group sequential design.
Normally this function will be passed to \code{gsDesign()} in the parameter 
\code{sfu} for the upper bound or
\code{sfl} for the lower bound to specify a spending function family for a design.
When passed to \code{gsDesign()}, the value of \code{param} would be passed to \code{sfLinear} through the \code{gsDesign()} arguments \code{sfupar} for the upper bound and \code{sflpar} for the lower bound.
\end{Description}
\begin{Usage}
\begin{verbatim}
sfLinear(alpha, t, param)
\end{verbatim}
\end{Usage}
\begin{Arguments}
\begin{ldescription}
\item[\code{alpha}] Real value \eqn{> 0}{} and no more than 1. Normally, 
\code{alpha=0.025} for one-sided Type I error specification
or \code{alpha=0.1} for Type II error specification. However, this could be set to 1 if for descriptive purposes you wish to see the proportion of spending as a function of the proportion of sample size or information.
\item[\code{t}] A vector of points with increasing values from 0 to 1, inclusive. Values of the proportion of 
sample size or information for which the spending function will be computed.
\item[\code{param}] A vector with a positive, even length. All values must be strictly between 0 and 1.
Letting \code{m <- length(param/2)}, the first \code{m} points in param specify increasing values strictly between 0 and 1 where the proportion of total spending is specified. 
The last \code{m} points in param specify increasing values strictly between 0 and 1 with the cumulative proportion of spending at the timepoints in the first part of the vector.
\end{ldescription}
\end{Arguments}
\begin{Value}
An object of type \code{spendfn}. 
The cumulative spending returned in \code{sfLinear\$spend} is 0 for \code{t=0} and \code{alpha} for \code{t>=1}. 
For \code{t} between specified points, linear interpolation is used to determine \code{sfLinear\$spend}. 
See \code{\LinkA{Spending function overview}{Spending function overview}} for further details.
\end{Value}
\begin{Note}\relax
The manual is not linked to this help file, but is available in library/gsdesign/doc/gsDesignManual.pdf
in the directory where R is installed.
\end{Note}
\begin{Author}\relax
Keaven Anderson \email{keaven\_anderson@merck.}
\end{Author}
\begin{References}\relax
Jennison C and Turnbull BW (2000), \emph{Group Sequential Methods with Applications to Clinical Trials}.
Boca Raton: Chapman and Hall.
\end{References}
\begin{SeeAlso}\relax
\LinkA{Spending function overview}{Spending function overview}, \code{\LinkA{gsDesign}{gsDesign}}, \LinkA{gsDesign package overview}{gsDesign package overview}
\end{SeeAlso}
\begin{Examples}
\begin{ExampleCode}
# set up alpha spending and beta spending to be piecewise linear
sfupar <- c(.2, .4, .05, .2)
sflpar <- c(.3, .5, .65, .5, .75, .9)
x <- gsDesign(sfu=sfLinear, sfl=sfLinear, sfupar=sfupar, sflpar=sflpar)
plot(x, plottype="sf")
x

\end{ExampleCode}
\end{Examples}

