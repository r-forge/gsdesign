\section{Acknowledgements}

I would first like to thank Dan (Jennifer) Sun and Zhongxin (John) Zhang. Jennifer worked as a summer intern with John and me in the summer of 2006. Her project was to take some C programs I had written to derive group sequential designs and build an R package around them. Together, she and I wrote the first version of the gsDesign package and, with John, its manual. The package has been used to help design several trials at Merck and I would like to thank Cong Chen, Zhiping (Linda) Sun, Yang Song, Weile He, Santosh Sutradhar, Yanli Zhao, Jason Clark, Yevgen Tymofyeyev and Fang Liu for their interest in applying the methods from the package and/or for reviewing the manual. Shanhong Guan contributed the sample size function for survival endpoints. I apologize to those I have forgotten!

Near the end of 2008, gsDesign development received additional financial support as part of an effort to expand the use of adaptive design at Merck. Thanks to Jerry Schindler, Vikas Patel and Sumeet Bagga for their support of this effort. The support was used to strengthen gsDesign as an example of an open source effort to develop high quality software. REvolution Computing was contracted to work on updating the package by reviewing code and documentation as well as creating a testing package to be used in evaluation of the quality of the programming as well as for installation qualification. They ended up doing much more. Rich Calaway has done substantial editing of the documentation and recommended other revisions, Kellie Wills performed much of the stress testing, and Sue Ranney assembled the installation qualification document. Chris Wiggins was my first contact at REvolution; he and I share common views about open source R packages. Finally, I would like to thank William Constantine who was my primary (nearly daily) contact for this project. Bill completely updated the error handling in gsDesign, set up a tremendous amount of the installation qualification testing and substantially reformatted the R code as well as the text markup for the manual. I look forward to any future potential collaboration with REvolution as the experience was quite productive and pleasant.

Keaven Anderson
