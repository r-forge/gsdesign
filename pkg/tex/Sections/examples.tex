\subsubsection{Using standard spending functions}
Next we consider some simple alternatives to the default spending function parameters.
Spending functions are covered in greater detail in Section \ref{sec:spendfn}. 
In the first code line following, we replace lower and upper spending function parameters with $1$ and $-2$, respectively; the default Hwang-Shih-DeCani spending function family is still used. 
In the second line, we specify a Kim-DeMets (power) spending function for
both the lower bound (with the parameters \texttt{sfl=sfPower} and \texttt{sflpar=2}) and the upper bounds (with the parameters \texttt{sfu=sfPower} and \texttt{sfupar=3}).
Then we compare bounds from the three designs. 
Bounds for the power spending function design are quite comparable to the default design. Generally, choosing between these two spending function families is somewhat arbitrary. The alternate Hwang-Shih-DeCani design uses more aggressive stopping boundaries. The last lines below show that sample size inflation from a fixed design is about 25\% for the the design with more aggressive stopping boundaries compared to about 7\% for each of the other designs. 

\bigskip

\begin{verbatim}
> x <- gsDesign()
> xHSDalt <- gsDesign(sflpar=1, sfupar=-2)
> xKD <- gsDesign(sfl=sfPower, sflpar=2, sfu=sfPower, sfupar=3)
> x$upper$bound
[1] 3.010739 2.546531 1.999226
> xHSDalt$upper$bound
[1] 2.677524 2.385418 2.063740
> xKD$upper$bound
[1] 3.113017 2.461933 2.008705
> x$lower$bound
[1] -0.2387240  0.9410673  1.9992264
> xHSDalt$lower$bound
[1] 0.3989132 1.3302944 2.0637399
> xKD$lower$bound
[1] -0.3497491  0.9822541  2.0087052
> x$n.I[3]
[1] 1.069883
> xHSDalt$n.I[3]
[1] 1.254268
> xKD$n.I[3]
[1] 1.071011
>\end{verbatim}

\subsubsection{Two-sided testing: the diabetes trial}
