\section{Installation and Online Help\label{sec:install}}

The package comes in a binary format for a Windows platform in the file
gsDesign-1.3.zip (may be updated to fix bugs in a file such as
gsDesign-1.3.01.zip). This file includes a copy of this manual in the file
gsDesignManual.pdf. The source, available for other platforms (but only tested
minimally!) is in the file gsDesign-1.3.tar.gz (may be updated to fix bugs in
a file such as gsDesign-1.3.01.tar.gz). Following are basic instructions for
installing the binary version on a Windows machine. It is assumed that a
`recent' version of R is installed. From the Windows interface of R, select
the Packages menu line and from this menu select Install packages from local
zip files\ldots. Browse to select gsDesign-1.3.zip. Once installed, you need
to load the package by selecting the Packages menu line, selecting Load
package\ldots\ from this menu, and then selecting gsDesign. You are now ready
to use the routines in the package. The most up-to-date version of this manual
and the code is also available at \texttt{http://r-forge.r-project.org}.

\bigskip

Online help can be obtained by entering the following on the command line:

\bigskip

\texttt{%
%TCIMACRO{\TEXTsymbol{>} }%
%BeginExpansion
$>$
%EndExpansion
help(gsDesign)}

\bigskip

There are many help topics covered there which should be sufficient
information to keep you from needing to use this document for day-to-day use.
In the Window version of R, this brings up a "Contents" window and a
documentation window. In the contents window, open the branch of documentation
headed by "Package gsDesign: Titles." The help files are organized
sequentially under this heading. 

