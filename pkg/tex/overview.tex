\section{Overview}

Three R functions are supplied to provide basic computations related to
designing group sequential clinical trials:

\begin{enumerate}
\item The \texttt{gsDesign()} function provides sample size and
boundaries for a group sequential design based on treatment effect, spending
functions for boundary crossing probabilities, and relative timing of each
analysis. Standard and user-specified spending functions may be used. In
addition to spending function designs, the family of Wang-Tsiatis
designs---including O'Brien-Fleming and Pocock designs---are also available.

\item The \texttt{gsProbability()} function computes boundary crossing 
probabilities and expected sample size of a design for arbitrary 
user-specified treatment effects, bounds, and interim analysis sample sizes.

\item The \texttt{gsCP()} function computes the conditional probability of 
future boundary crossing given a result at an interim analysis. 
The \texttt{gsCP()} function returns a value of the same type as 
\texttt{gsProbability()}.
\end{enumerate}

The package design strategy should make these routines useful both as an
everyday tool for simple group sequential design as well as a research tool
for a wide variety of group sequential design problems. Both \texttt{print()}
and \texttt{plot()} functions are available for both \texttt{gsDesign()} and
\texttt{gsProbability()}. This should make it easy to incorporate design
specification and properties into documents, as required.

Functions are set up to be called directly from the R command line. Default
arguments and output for \texttt{gsDesign()} are included to make initial use
simple. Sufficient options are available, however, to make the routine very
flexible. 

To get started with \texttt{gsDesign}, read Section~\ref{sec:basicfeatures}, 
Basic Features,  and then proceed to Section~\ref{sec:detailedex}, Detailed 
Examples, to get a feel for how the routines work. To try the routines out, 
read Section~\ref{sec:install}, Installation and Online Help. These three
sections along with online help allow you to develop a design
quickly without reading the full specification given in succeeding sections.
For those interested in the theory behind this package, 
Section~\ref{sec:statmethods}, Statistical Methods, provides background.

Complete clean-up and review of the manual (e.g., adding the references) will
occur in version 1.20. Generally, the manual should be up-to-date with package revisions.

