\HeaderA{nSurvival}{3.3: Time-to-event sample size calculation (Lachin-Foulkes)}{nSurvival}
\aliasA{Survival sample size}{nSurvival}{Survival sample size}
\keyword{design}{nSurvival}
\begin{Description}\relax
\code{nSurvival()} is used to calculate the sample size for a clinical trial with a time-to-event endpoint. The Lachin and Foulkes (1986) method is used.
\end{Description}
\begin{Usage}
\begin{verbatim}
nSurvival(lambda.0, lambda.1, Ts, Tr, eta = 0, rand.ratio = 1,
      alpha = 0.05, beta = 0.10, sided = 2, approx = FALSE,
      type = c("rr", "rd"), entry = c("unif", "expo"), gamma = NA)
\end{verbatim}
\end{Usage}
\begin{Arguments}
\begin{ldescription}
\item[\code{lambda.0, lambda.1}] event hazard rate for placebo and treatment
group respectively.
\item[\code{eta}] equal dropout hazard rate for both groups.
\item[\code{rand.ratio}] randomization ratio between placebo and treatment
group. Default is balanced design, i.e., randomization ratio is 1.
\item[\code{Ts}] maximum study duration.
\item[\code{Tr}] accrual (recruitment) duration.
\item[\code{alpha}] type I error rate. Default is 0.05 since 2-sided testing is default.
\item[\code{beta}] type II error rate. Default is 0.10 (90\% power).
\item[\code{sided}] one or two-sided test? Default is two-sided test.
\item[\code{approx}] logical. If \code{TRUE}, the approximation sample size
formula for risk difference is used.
\item[\code{type}] type of sample size calculation: risk ratio (\dQuote{rr}) or risk
difference (\dQuote{rd}).
\item[\code{entry}] patient entry type: uniform entry (\code{"unif"}) or exponential
entry (\code{"expo"}).
\item[\code{gamma}] rate parameter for exponential entry. \code{NA} if entry type is
\code{"unif"} (uniform). A non-zero value is supplied if entry type is
\code{"expo"} (exponential).
\end{ldescription}
\end{Arguments}
\begin{Details}\relax
\code{nSurvival} produces the number of subjects and events for a set of
pre-specified trial parameters, such as accrual duration and follow-up
period. The calculation is based on Lachin and Foulkes method and can
be used for risk ratio or risk difference. The function also consider
non-uniform entry as well as uniform entry. 

If the logical approx is \code{TRUE}, the variance under alternative
hypothesis is used to replace the variance under null hypothesis.

For non-uniform entry. a non-zero value of gamma for exponential entry
must be supplied. For positive gamma, the entry distribution is
convex, whereas for negative gamma, the entry distribution is concave.
\end{Details}
\begin{Value}
\code{nSurvival} produces a list with the following component returned:
\begin{ldescription}
\item[\code{Method}] As input.
\item[\code{Entry}] As input.
\item[\code{Sample.size}] Number of subjects.
\item[\code{Num.events}] Number of events.
\item[\code{Hazard.p, Hazard.t}] hazard rate for placebo and treatment group. As input.
\item[\code{Dropout}] as input.
\item[\code{Frac.p, Frac.t}] randomization proportion for placebo and
treatment. As input.
\item[\code{Gamma}] as input.
\item[\code{Alpha}] as input.
\item[\code{Beta}] as input.
\item[\code{Sided}] as input.
\item[\code{Study.dura}] Study duration.
\item[\code{Accrual}] Accrual period.
\end{ldescription}
\end{Value}
\begin{Author}\relax
Shanhong Guan \email{shanhong\_guan@merck.com}
\end{Author}
\begin{References}\relax
Lachin JM and Foulkes MA (1986),
Evaluation of Sample Size and Power for Analyses of Survival
with Allowance for Nonuniform Patient Entry, Losses to Follow-Up,
Noncompliance, and Stratification. \emph{Biometrics}, 42, 507-519.
\end{References}
\begin{Examples}
\begin{ExampleCode}

# consider a trial with 
# 2 year maximum follow-up
# 6 month uniform enrollment
# Treatment/placebo hazards = 0.1/0.2 per 1 person-year
# drop out hazard 0.1 per 1 person-year
# alpha = 0.05 (two-sided)
# power = 0.9 (default beta=.1)

ss <- nSurvival(lambda.0=.2 , lambda.1=.1, eta = .1, Ts = 2, Tr = .5,
                sided=1, alpha=.025)

#  symmetric, 2-sided design with O'Brien-Fleming-like boundaries
#  sample size is computed based on a fixed design requiring n=100
        x<-gsDesign(k = 5, test.type = 2)
        x
# boundary plot
        plot(x)
# power plot
        plot(x, plottype = 2)
# total sample size
   ceiling(x$n.I[x$k] * ss$Sample.size)
# number of events at analyses
   ceiling(ss$Num.events * x$n.I)
\end{ExampleCode}
\end{Examples}

