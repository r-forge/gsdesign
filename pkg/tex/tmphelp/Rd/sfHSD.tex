\HeaderA{sfHSD}{4.1: Hwang-Shih-DeCani Spending Function}{sfHSD}
\keyword{design}{sfHSD}
\begin{Description}\relax
The function \code{sfHSD} implements a Hwang-Shih-DeCani spending function.
This is the default spending function for \code{gsDesign()}.
Normally it will be passed to \code{gsDesign} in the parameter \code{sfu} for the upper bound or
\code{sfl} for the lower bound to specify a spending function family for a design.
In this case, the user does not need to know the calling sequence.
The calling sequence is useful, however, when the user wishes to plot a spending function as demonstrated below
in examples.
\end{Description}
\begin{Usage}
\begin{verbatim}
sfHSD(alpha, t, param)
\end{verbatim}
\end{Usage}
\begin{Arguments}
\begin{ldescription}
\item[\code{alpha}] Real value \eqn{> 0}{} and no more than 1. Normally, 
\code{alpha=0.025} for one-sided Type I error specification or \code{alpha=0.1} for Type II error specification. However, this could be set to 1 if for descriptive purposes you wish to see the proportion of spending as a function of the proportion of sample size/information.
\item[\code{t}] A vector of points with increasing values from 0 to 1, inclusive. Values of the proportion of sample size/information for which the spending function will be computed.
\item[\code{param}] A single real value specifying the gamma parameter for which Hwang-Shih-DeCani spending is to be computed; allowable range is [-40, 40]
\end{ldescription}
\end{Arguments}
\begin{Details}\relax
A Hwang-Shih-DeCani spending function takes the form
\deqn{f(t;\alpha, \gamma)=\alpha(1-e^{-\gamma t})/(1-e^{-\gamma})}{f(t; alpha, gamma) = alpha * (1-exp(-gamma * t))/(1 - exp(-gamma))}
where \eqn{\gamma}{gamma} is the value passed in \code{param}.
A value of \eqn{\gamma=-4}{gamma=-4} is used to approximate an O'Brien-Fleming design (see \code{\LinkA{sfExponential}{sfExponential}} for a better fit), 
while a value of \eqn{\gamma=1}{gamma=1} approximates a Pocock design well.
\end{Details}
\begin{Value}
An object of type \code{spendfn}. See \LinkA{Spending function overview}{Spending function overview} for further details.
\end{Value}
\begin{Note}\relax
The manual is not linked to this help file, but is available in library/gsdesign/doc/gsDesignManual.pdf
in the directory where R is installed.
\end{Note}
\begin{Author}\relax
Keaven Anderson \email{keaven\_anderson@merck.com}
\end{Author}
\begin{References}\relax
Jennison C and Turnbull BW (2000), \emph{Group Sequential Methods with Applications to Clinical Trials}.
Boca Raton: Chapman and Hall.
\end{References}
\begin{SeeAlso}\relax
\LinkA{Spending function overview}{Spending function overview}, \code{\LinkA{gsDesign}{gsDesign}}, \LinkA{gsDesign package overview}{gsDesign package overview}
\end{SeeAlso}
\begin{Examples}
\begin{ExampleCode}
# design a 4-analysis trial using a Hwang-Shih-DeCani spending function 
# for both lower and upper bounds 
x <- gsDesign(k=4, sfu=sfHSD, sfupar=-2, sfl=sfHSD, sflpar=1)

# print the design
x

# since sfHSD is the default for both sfu and sfl,
# this could have been written as
x <- gsDesign(k=4, sfupar=-2, sflpar=1)

# print again
x

# plot the spending function using many points to obtain a smooth curve
# show default values of gamma to see how the spending function changes
# also show gamma=1 which is supposed to approximate a Pocock design
t <- 0:100/100
plot(t,  sfHSD(0.025, t, -4)$spend,
   xlab="Proportion of final sample size", 
   ylab="Cumulative Type I error spending", 
   main="Hwang-Shih-DeCani Spending Function Example", type="l")
lines(t, sfHSD(0.025, t, -2)$spend, lty=2)
lines(t, sfHSD(0.025, t, 1)$spend, lty=3)
legend(x=c(.0, .375), y=.025*c(.8, 1), lty=1:3, 
    legend=c("gamma= -4", "gamma= -2", "gamma= 1"))
\end{ExampleCode}
\end{Examples}

