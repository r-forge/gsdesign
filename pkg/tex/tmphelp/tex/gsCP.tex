\HeaderA{gsCP}{2.4: Conditional Power Computation}{gsCP}
\keyword{design}{gsCP}
\begin{Description}\relax
\code{gsCP()} takes a given group sequential design, assumes an interim z-statistic 
at a specified interim analysis and computes boundary crossing probabilities at future planned analyses.
\end{Description}
\begin{Usage}
\begin{verbatim}
gsCP(x, theta=NULL, i=1, zi=0, r=18)
\end{verbatim}
\end{Usage}
\begin{Arguments}
\begin{ldescription}
\item[\code{x}] An object of type \code{gsDesign} or \code{gsProbability}
\item[\code{theta}] \eqn{\theta}{theta} value(s) at which conditional power is to be computed; if \code{NULL}, 
an estimated value of \eqn{\theta}{theta} based on the interim test statistic (\code{zi/sqrt(x\$n.I[i])}) as well as at \code{x\$theta}
is computed.
\item[\code{i}] analysis at which interim z-value is given
\item[\code{zi}] interim z-value at analysis i (scalar)
\item[\code{r}] Integer value controlling grid for numerical integration as in Jennison and Turnbull (2000); 
default is 18, range is 1 to 80. 
Larger values provide larger number of grid points and greater accuracy.
Normally \code{r} will not be changed by the user.
\end{ldescription}
\end{Arguments}
\begin{Details}\relax
See Conditional power section of manual for further clarification. See also Muller and Schaffer (2001) for background theory.
\end{Details}
\begin{Value}
An object of the class \code{gsProbability}.
Based on the input design and the interim test statistic, the output object has bounds for test statistics
computed based on observations after interim \code{i} that are equivalent to the original design crossing boundaries conditional
on the interim test statistic value input. 
Boundary crossing probabilities are computed for the input 
\eqn{\theta}{theta} values.
\end{Value}
\begin{Note}\relax
The manual is not linked to this help file, but is available in library/gsdesign/doc/gsDesignManual.pdf in the directory where R is installed.
\end{Note}
\begin{Author}\relax
Keaven Anderson \email{keaven\_anderson@merck.}
\end{Author}
\begin{References}\relax
Jennison C and Turnbull BW (2000), \emph{Group Sequential Methods with Applications to Clinical Trials}.
Boca Raton: Chapman and Hall.

Muller, Hans-Helge and Schaffer, Helmut (2001), Adaptive group sequential designs for clinical trials:
combining the advantages of adaptive and classical group sequential approaches. \emph{Biometrics};57:886-891.
\end{References}
\begin{SeeAlso}\relax
\code{\LinkA{gsDesign}{gsDesign}}, \code{\LinkA{gsProbability}{gsProbability}}, \code{\LinkA{gsBoundCP}{gsBoundCP}}
\end{SeeAlso}
\begin{Examples}
\begin{ExampleCode}
# set up a group sequential design
x <- gsDesign(k=5)
x

# assuming a z-value of .5 at analysis 2, what are conditional 
# boundary crossing probabilities for future analyses
# assuming theta values from x as well as a value based on the interim
# observed z
CP <- gsCP(x, i=2, zi=.5)
CP

# summing values for crossing future upper bounds gives overall
# conditional power for each theta value
CP$theta
CP$upper$prob 
\end{ExampleCode}
\end{Examples}

