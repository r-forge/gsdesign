\HeaderA{gsBound}{2.6: Boundary derivation - low level}{gsBound}
\aliasA{gsBound1}{gsBound}{gsBound1}
\keyword{design}{gsBound}
\begin{Description}\relax
\code{gsBound()} and \code{gsBound1()} are lower-level functions used to find boundaries for a group sequential design.
They are not recommended (especially \code{gsBound1()}) for casual users.
These functions do not adjust sample size as \code{gsDesign()} does to ensure appropriate power for a design.

\code{gsBound()} computes upper and lower bounds given boundary crossing probabilities assuming a mean of 0, the usual null hypothesis.
\code{gsBound1()} computes the upper bound given a lower boundary, upper boundary crossing probabilities and an arbitrary mean (\code{theta}).
\end{Description}
\begin{Usage}
\begin{verbatim}
gsBound(I, trueneg, falsepos, tol=0.000001, r=18)
gsBound1(theta, I, a, probhi, tol=0.000001, r=18, printerr=0)
\end{verbatim}
\end{Usage}
\begin{Arguments}
Note that all vector arguments should have the same length which will be denoted here as \code{k}.
\begin{ldescription}
\item[\code{theta}] Scalar containing mean (drift) per unit of statistical information.
\item[\code{I}] Vector containing statistical information planned at each analysis.
\item[\code{a}] Vector containing lower bound that is fixed for use in \code{gsBound1}.
\item[\code{trueneg}] Vector of desired probabilities for crossing upper bound assuming mean of 0.
\item[\code{falsepos}] Vector of desired probabilities for crossing lower bound assuming mean of 0.
\item[\code{probhi}] Vector of desired probabilities for crossing upper bound assuming mean of theta.
\item[\code{tol}] Tolerance for error (scalar; default is 0.000001). Normally this will not be changed by the user.
This does not translate directly to number of digits of accuracy, so use extra decimal places.
\item[\code{r}] Single integer value controlling grid for numerical integration as in Jennison and Turnbull (2000); 
default is 18, range is 1 to 80. 
Larger values provide larger number of grid points and greater accuracy.
Normally \code{r} will not be changed by the user.
\item[\code{printerr}] If this scalar argument set to 1, this will print messages from underlying C program.
Mainly intended to notify user when an output solution does not match input specifications.
This is not intended to stop execution as this often occurs when deriving a design in \code{gsDesign}
that uses beta-spending.
\end{ldescription}
\end{Arguments}
\begin{Details}\relax
The function \code{gsBound1()} requires special attention to detail and knowledge of behavior when a design corresponding to the input parameters does not exist.
\end{Details}
\begin{Value}
Both routines return a list. Common items returned by the two routines are: 
\begin{ldescription}
\item[\code{k}] The length of vectors input; a scalar.
\item[\code{theta}] As input in \code{gsBound1()}; 0 for \code{gsBound()}.
\item[\code{I}] As input.
\item[\code{a}] For \code{gsbound1}, this is as input. For \code{gsbound} this is the derived lower boundary required to yield the input boundary crossing probabilities under the null hypothesis.
\item[\code{b}] The derived upper boundary required to yield the input boundary crossing probabilities under the null hypothesis.
\item[\code{tol}] As input.
\item[\code{r}] As input.
\item[\code{error}] Error code. 0 if no error; greater than 0 otherwise.
\item[\code{rates}] a list containing two items:
\item[\code{falsepos}] vector of upper boundary crossing probabilities as input.
\item[\code{trueneg}] vector of lower boundary crossing probabilities as input.
\item[\code{problo}] vector of lower boundary crossing probabilities; computed using input lower bound and derived upper bound.
\item[\code{probhi}] vector of upper boundary crossing probabilities as input.
\end{ldescription}
\end{Value}
\begin{Note}\relax
The manual is not linked to this help file, but is available in library/gsdesign/doc/gsDesignManual.pdf
in the directory where R is installed.
\end{Note}
\begin{Author}\relax
Keaven Anderson \email{keaven\_anderson@merck.}
\end{Author}
\begin{References}\relax
Jennison C and Turnbull BW (2000), \emph{Group Sequential Methods with Applications to Clinical Trials}.
Boca Raton: Chapman and Hall.
\end{References}
\begin{SeeAlso}\relax
\LinkA{gsDesign package overview}{gsDesign package overview}, \code{\LinkA{gsDesign}{gsDesign}}, \code{\LinkA{gsProbability}{gsProbability}}
\end{SeeAlso}
\begin{Examples}
\begin{ExampleCode}
# set boundaries so that probability is .01 of first crossing
# each upper boundary and .02 of crossing each lower boundary
# under the null hypothesis
x <- gsBound(I=c(1, 2, 3)/3, trueneg=array(.02, 3),
             falsepos=array(.01, 3))
x

#  use gsBound1 to set up boundary for a 1-sided test
x <- gsBound1(theta= 0, I=c(1, 2, 3) / 3, a=array(-20, 3),
              probhi=c(.001, .009, .015))
x$b

# check boundary crossing probabilities with gsProbability 
y <- gsProbability(k=3, theta=0, n.I=x$I, a=x$a, b=x$b)$upper$prob

#  Note that gsBound1 only computes upper bound 
#  To get a lower bound under a parameter value theta:
#      use minus the upper bound as a lower bound
#      replace theta with -theta
#      set probhi as desired lower boundary crossing probabilities 
#  Here we let set lower boundary crossing at 0.05 at each analysis
#  assuming theta=2.2 
y <- gsBound1(theta=-2.2, I=c(1, 2, 3)/3, a= -x$b, 
              probhi=array(.05, 3))
y$b

#  Now use gsProbability to look at design
#  Note that lower boundary crossing probabilities are as
#  specified for theta=2.2, but for theta=0 the upper boundary
#  crossing probabilities are smaller than originally specified
#  above after first interim analysis
gsProbability(k=length(x$b), theta=c(0, 2.2), n.I=x$I, b=x$b, a= -y$b)
\end{ExampleCode}
\end{Examples}

