\HeaderA{Wang-Tsiatis Bounds}{5.0: Wang-Tsiatis Bounds}{Wang.Rdash.Tsiatis Bounds}
\aliasA{O'Brien-Fleming Bounds}{Wang-Tsiatis Bounds}{O'Brien.Rdash.Fleming Bounds}
\aliasA{Pocock Bounds}{Wang-Tsiatis Bounds}{Pocock Bounds}
\keyword{design}{Wang-Tsiatis Bounds}
\begin{Description}\relax
\code{gsDesign} offers the option of using Wang-Tsiatis bounds as an alternative to 
the spending function approach to group sequential design.
Wang-Tsiatis bounds include both Pocock and O'Brien-Fleming designs.
Wang-Tsiatis bounds are currently only available for 1-sided and symmetric 2-sided designs.
Wang-Tsiatis bounds are typically used with equally spaced timing between analyses, but
the option is available to use them with unequal spacing.
\end{Description}
\begin{Details}\relax
Wang-Tsiatis bounds are defined as follows.
Assume \eqn{k}{} analyses and let \eqn{Z_i}{} represent the upper bound and \eqn{t_i}{} the proportion of the
total planned sample size for the \eqn{i}{}-th analysis, 
\eqn{i=1,2,\ldots,k}{}.
Let \eqn{\Delta}{Delta} be a real-value. 
Typically \eqn{\Delta}{Delta} will range from 0 (O'Brien-Fleming design) to 0.5 (Pocock design).
The upper boundary is defined by 
\deqn{ct_i^{\Delta-0.5}}{}
for \eqn{i= 1,2,\ldots,k}{} where \eqn{c}{} depends on the other parameters.
The parameter \eqn{\Delta}{Delta} is supplied to \code{gsDesign()} in the parameter \code{sfupar}.
For O'Brien-Fleming and Pocock designs there is also a calling sequence that does not require a parameter.
See examples.
\end{Details}
\begin{Note}\relax
The manual is not linked to this help file, but is available in library/gsdesign/doc/gsDesignManual.pdf
in the directory where R is installed.
\end{Note}
\begin{Author}\relax
Keaven Anderson \email{keaven\_anderson@merck.}
\end{Author}
\begin{References}\relax
Jennison C and Turnbull BW (2000), \emph{Group Sequential Methods with Applications to Clinical Trials}.
Boca Raton: Chapman and Hall.
\end{References}
\begin{SeeAlso}\relax
\code{\LinkA{Spending function overview}{Spending function overview}, \LinkA{Spending function overview}{Spending function overview}}, \code{\LinkA{gsProbability}{gsProbability}}
\end{SeeAlso}
\begin{Examples}
\begin{ExampleCode}
# Pocock design
gsDesign(test.type=2, sfu="Pocock")

# alternate call to get Pocock design specified using 
# Wang-Tsiatis option and Delta=0.5
gsDesign(test.type=2, sfu="WT", sfupar=0.5)

# this is how this might work with a spending function approach
# Hwang-Shih-DeCani spending function with gamma=1 is often used 
# to approximate Pocock design
gsDesign(test.type=2, sfu=sfHSD, sfupar=1)

# unequal spacing works,  but may not be desirable 
gsDesign(test.type=2, sfu="Pocock", timing=c(.1, .2))

# spending function approximation to Pocock with unequal spacing 
# is quite different from this
gsDesign(test.type=2, sfu=sfHSD, sfupar=1, timing=c(.1, .2))

# One-sided O'Brien-Fleming design
gsDesign(test.type=1, sfu="OF")

# alternate call to get O'Brien-Fleming design specified using 
# Wang-Tsiatis option and Delta=0
gsDesign(test.type=1, sfu="WT", sfupar=0)
\end{ExampleCode}
\end{Examples}

