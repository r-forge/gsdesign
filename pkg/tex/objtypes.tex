\section{The gsDesign and gsProbability Object Types\label{sec:objtypes}}

This package creates three object (storage) classes to store the structured
information returned by \texttt{gsDesign()}, \texttt{gsProbability()} and
spending function routines. Thus, a single variable name can be used to
access, print or plot a set of returned values. \bigskip

\subsection{The spendfn Class\label{sec:spendfn}}

\bigskip

The first class, \texttt{spendfn}, is used to store information about either 
the upper or the lower bound of a design: the spending function, z-values for 
the stopping bounds, spending at each analysis and boundary crossing 
probabilities under parameter values of interest. The \texttt{spendfn} class
contains the following members:

\begin{itemize}
\item \texttt{sf}: a spending function or character value. Normally this
is a spending function, but if Wang-Tsiatis ("WT"), Pocock ("Pocock"), or
O'Brien-Fleming ("OF") are specified (these are not spending function
designs), then \texttt{sf} is a character string. This is useful if you
later wish to plot the spending function for a larger set of values than
specified in an original call. ({\bf Note for writers of new spending 
functions:} this is not set in the spending function itself, but in 
\texttt{gsDesign()})

\item \texttt{name}: a character string specifying the spending function used.

\item \texttt{param}: parameter value or values to fully specify the
spending function using the spending function family specified in \texttt{sf}.

\item \texttt{parname}: a text string or vector of text strings (matching
the length of \texttt{param}) supplying the names of parameters.

\item \texttt{spend}: a vector containing the amount of $\alpha$- or $\beta
$-spending at each analysis; this is determined in a call to
\texttt{gsDesign()}.

\item \texttt{bound}: this is null when returned from the spending function,
but is set in \texttt{gsDesign()} if the spending function is called from
there. Contains z-values for bounds of a design.

\item \texttt{prob}: this is null when returned from the spending function,
but is set in \texttt{gsDesign()} if the spending function is called from
there. Contains a \texttt{k}$\times$\texttt{n} matrix of probabilities of
boundary crossing at\texttt{\ i}-th analysis for \texttt{j}-th \texttt{theta}
value in \texttt{prob[i,j]}. Upon return from \texttt{gsDesign()},
\texttt{n}=2 and the values of theta considered are 0 and \texttt{delta}. More
values of \texttt{theta} can be added by a call to \texttt{gsProbability()}.
Columns correspond to the values specified by \texttt{theta}. All boundary
crossing is assumed to be binding for this computation; that is, the trial
must stop if a boundary is crossed.

\item \texttt{errcode}: 0 if no error in spending function specification,
%TCIMACRO{\TEXTsymbol{>} }%
%BeginExpansion
$>$
%EndExpansion
0 otherwise.

\item \texttt{errmsg}: a text string corresponding to \texttt{errcode}; for
no error, this is "No error."
\end{itemize}

\subsection{The gsProbability Class}

\bigskip

The second object class is \texttt{gsProbability}. The members of this class 
on output from \texttt{gsProbability()} are: \texttt{k}, \texttt{n.I}, 
\texttt{lower}, \texttt{upper}, \texttt{r}, \texttt{theta}, \texttt{errcode}, 
and \texttt{errmsg}. Most of these are described in the input to
\texttt{gsProbability()}. The exceptions are as follows:

\begin{itemize}
\item \texttt{upper}: on output, contains information on upper spending;
this is a member of the \texttt{spendfn} class described above. Values of
\texttt{upper} returning in a \texttt{gsProbability} class contain only the
elements \texttt{bound} and \texttt{prob}.

\item \texttt{lower}: on output, contains information on lower spending;
this is a member of the \texttt{spendfn} class described above. Values of
\texttt{lower} returning in a \texttt{gsProbability} class contain only the
elements \texttt{bound} and \texttt{prob}.

\item \texttt{en}: a vector of the same length as \texttt{theta} containing
expected sample sizes for the trial design corresponding to each value in the
vector \texttt{theta}.

\item \texttt{errcode}: 0 if no error detected in call to
\texttt{gsDesign()} or \texttt{gsProbability()},
%TCIMACRO{\TEXTsymbol{>} }%
%BeginExpansion
$>$
%EndExpansion
0 otherwise.

\item \texttt{errmsg}: a text string corresponding to \texttt{errcode}; for
no error, this is "No errors detected."
\end{itemize}

\subsection{The gsDesign Class}

The third and final object class is \texttt{gsDesign}. This inherits the class
\texttt{gsProbability} and, in addition, has the following members on output from
\texttt{gsDesign()}: \texttt{test.type}, \texttt{alpha}, \texttt{beta},
\texttt{delta}, \texttt{n.fix}, \texttt{timing}, \texttt{tol}, 
\texttt{maxn.IPlan}. In
addition, all elements of the class 
\text{spendfn} are returned in \texttt{upper} and
\texttt{lower}. Most of these variables are as described in the input to
\texttt{gsDesign()} or in the \texttt{spendfn} or \texttt{gsProbability} 
class description. The exceptions are as follows:

\begin{itemize}
\item \texttt{timing}: when this is input as 1, the output is transformed to
a vector of equally spaced analyses as follows: (1, 2, \ldots, \texttt{k}%
)/\texttt{k}. Otherwise, this should be a vector of length \texttt{k}, with
increasing values greater than 0 and the greatest value equal to 1. A value of
\texttt{timing[2] = 0.4} is translated as the second interim analysis being performed
after 40\% of the total observations planned for the final analysis.

\item \texttt{delta}: the standardized effect size; if this was input as 0,
it is recomputed with the following formula:%
\[
\delta=\frac{\Phi^{-1}(1-\alpha)+\Phi^{-1}(1-\beta)}{\sqrt{N_{fix}}}
\]


where $\delta$=\texttt{delta}, $N_{fix}$=\texttt{n.fix}, $\alpha
$=\texttt{alpha}, and $\beta$=\texttt{beta}.

\item \texttt{n.I}: this is the sample size or information required at each
analysis, depending on how \texttt{gsDesign()} is called. In addition to the
following descriptions, see Section~\ref{sec:detailedex}, Detailed Examples.

\begin{enumerate}
\item If \texttt{n.I} was input, the same value remains on output

\item If input to \texttt{gsDesign()} was \texttt{n.fix} = 1 and
\texttt{delta}=0, this results in returning a sample-size multiplier in
\texttt{n.I} that can be used to set sample size at interim analyses according
to the sample size required for a fixed design without interim analysis. That
is, use a formula for a fixed design that matches the type of data you are
collecting, and multiply this fixed design sample size by \texttt{n.I} to
obtain sample sizes at interim and final analyses for the desired group
sequential design.

\item Similarly, if you have a fixed design sample size which you wish to
modify using \texttt{gsDesign()} to obtain an appropriate group sequential
design, input \texttt{n.fix} as the fixed sample size design and input
\texttt{delta}=0. In this case \texttt{n.I} returns with sample sizes for
each analysis in the group sequential design.
\end{enumerate}

\item \texttt{upper}, \texttt{lower}: these are output as class 
\texttt{spendfn}, as described above. All elements of the 
\texttt{spendfn} class are returned when these are
generated from a call to \texttt{gsDesign()}.
\end{itemize}

